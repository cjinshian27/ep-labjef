% arara: pdflatex

\documentclass{article}

\usepackage{float}
\restylefloat{table}

\usepackage[table]{xcolor}
\colorlet{lightgray}{gray!10}

\usepackage[margin=1cm]{geometry}
\begin{document}

\noindent
\textbf{PESSOA}
\vspace{-5pt}

\begin{table}[H]
  \renewcommand{\arraystretch}{1.5}
  \rowcolors{1}{lightgray}{lightgray}
  \begin{tabular}{|c|c|c|c|c|c|c|c|}
    \hline
    \underline{id} &
    cpf &
    nome & 
    endereco &
    senha &
    login &
    instituicao &
    data\_nascimento \\
    \hline
  \end{tabular}
\end{table}

\noindent
\textbf{PERFIL}
\vspace{-5pt}

\begin{table}[H]
  \renewcommand{\arraystretch}{1.5}
  \rowcolors{1}{lightgray}{lightgray}
  \begin{tabular}{|c|c|c|}
    \hline
    
    \underline{id} &
    codigo &
    tipo  \\
    \hline
  \end{tabular}
\end{table}

\noindent
\textbf{SERVIÇO}
\vspace{-5pt}

\begin{table}[H]
  \renewcommand{\arraystretch}{1.5}
  \rowcolors{1}{lightgray}{lightgray}
  \begin{tabular}{|c|c|c|c|c|}
    \hline
    \underline{id} &
    codigo &
    descricao &
    tipo &
    id\_perfil \\
    \hline
  \end{tabular}
\end{table}

\noindent
\textbf{DOCENTE}
\vspace{-5pt}

\begin{table}[H]
  \renewcommand{\arraystretch}{1.5}
  \rowcolors{1}{lightgray}{lightgray}
  \begin{tabular}{|c|c|c|c|}
    \hline
    \underline{id} &
    cpf\_docente &
    especialidade &
    funcao\_tecnica \\
    \hline
  \end{tabular}
\end{table}

\noindent
\textbf{FUNCIONARIO}
\vspace{-5pt}

\begin{table}[H]
  \renewcommand{\arraystretch}{1.5}
  \rowcolors{1}{lightgray}{lightgray}
  \begin{tabular}{|c|c|c|c|}
    \hline
    \underline{id} &
    cpf\_funcionario &
    especialidade &
    funcao\_tecnica \\
    \hline
  \end{tabular}
\end{table}

\noindent
\textbf{ALUNO}
\vspace{-5pt}

\begin{table}[H]
  \renewcommand{\arraystretch}{1.5}
  \rowcolors{1}{lightgray}{lightgray}
  \begin{tabular}{|c|c|c|c|}
    \hline
    \underline{id} &
    cpf\_aluno &
    curso &
    nota\_ingresso \\
    \hline
  \end{tabular}
\end{table}

\noindent
\textbf{DISCIPLINA}
\vspace{-5pt}

\begin{table}[H]
  \renewcommand{\arraystretch}{1.5}
  \rowcolors{1}{lightgray}{lightgray}
  \begin{tabular}{|c|c|c|c|c|}
    \hline
    \underline{id} &
    codigo &
    nome &
    ementa &
    data\_criacao \\
    \hline
  \end{tabular}
\end{table}

\noindent
\textbf{REL\_PESSOA\_PERFIL}
\vspace{-5pt}

\begin{table}[H]
  \renewcommand{\arraystretch}{1.5}
  \rowcolors{1}{lightgray}{lightgray}
  \begin{tabular}{|c|c|c|}
    \hline
    \underline{id} &
    id\_pessoa &
    id\_perfil \\
    \hline
  \end{tabular}
\end{table}

\noindent
\textbf{REL\_OFERECIMENTO}
\vspace{-5pt}

\begin{table}[H]
  \renewcommand{\arraystretch}{1.5}
  \rowcolors{1}{lightgray}{lightgray}
  \begin{tabular}{|c|c|c|c|c|c|c|}
    \hline
    \underline{id} &
    id\_docente &
    id\_aluno &
    id\_disciplina &
    nota\_obtida &
    data\_inicio &
    data\_fim \\
    \hline
  \end{tabular}
\end{table}

\end{document}
