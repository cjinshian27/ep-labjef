% arara: pdflatex

\documentclass[border=1cm]{standalone}
\usepackage[utf8]{inputenc}
\usepackage{graphicx}
\usepackage[portuguese]
{babel}

\usepackage{tikz}

\tikzstyle{isa} = [isosceles triangle, isosceles triangle apex angle=70, shape border rotate=270,
draw, black, thick, minimum size=3em]

\usetikzlibrary{er,positioning}

\begin{document}

\begin{tikzpicture}[auto,node distance=4cm]

% ------------------------------
%  Entidades e Relacionamentos
% ------------------------------

  \node[entity] (pessoa) {Pessoa};

  \node[relationship] (rel-pessoa-perfil) [right = 2cm of pessoa] {} edge (pessoa);

  \node[entity] (perfil) [right = 2cm of rel-pessoa-perfil] {Perfil} edge (rel-pessoa-perfil);

  \node[relationship] (rel-perfil-servico) [right = 2cm of perfil] {} edge (perfil);

  \node[entity] (servico) [right = 2cm of rel-perfil-servico] {Serviço} edge (rel-perfil-servico);

  \node[isa] (especializacao) [below right = 2 and 2 cm of pessoa] {} edge (pessoa);

  \node[entity] (funcionario) [below left = 5mm and 2cm of especializacao] {Funcionário} edge (especializacao);

  \node[entity] (aluno) [below = 2cm of especializacao] {Aluno} edge (especializacao);

  \node[entity] (docente) [below right = 5mm and 2cm of especializacao] {Docente} edge (especializacao);

  \node[relationship] (rel-oferecimento) [right = 2cm of aluno] {} edge (docente) edge (aluno);

  \node[entity] (disciplina) [right = 2cm of rel-oferecimento] {Disciplina} edge (rel-oferecimento);
  
% ---------------------
%    CARDINALIDADES
% ---------------------
  
  \path (rel-oferecimento) edge node [left] {$n$} (docente);
  \path (rel-oferecimento) edge node [above] {$m$} (disciplina);
  \path (rel-oferecimento) edge node [above] {$o$} (aluno);

  \path (rel-pessoa-perfil) edge node [above] {$n$} (pessoa);
  \path (rel-pessoa-perfil) edge node [above] {$m$} (perfil);

  \path (rel-perfil-servico) edge node [above] {$1$} (perfil);
  \path (rel-perfil-servico) edge node [above] {$m$} (servico);
   
  % \path (especializacao) edge node [above right] {\{Parcial\}} (pessoa);

% ---------------------
%        PESSOA
% ---------------------

  \node[attribute] (data-nascimento) [above = 2cm of pessoa] {Data nascimento} edge (pessoa);
  \node[attribute] (nome) [above left = 1cm and 0 of pessoa] {Nome} edge (pessoa);
  \node[attribute] (cpf) [above right = 1cm and 0 of pessoa] {CPF} edge (pessoa);
  \node[attribute] (endereco) [above left = 3mm and 1.2cm of pessoa] {Endereço} edge (pessoa);
  \node[attribute] (instituicao) [left = 1cm of pessoa] {Instituição} edge (pessoa);
  \node[attribute] (id-pessoa) [below = 1cm of pessoa] {\underline{Id}} edge (pessoa);
  \node[attribute] (login) [below left = 1cm and 0 of pessoa] {Login} edge (pessoa);
  \node[attribute] (senha) [below left = 3mm and 1.2cm of pessoa] {Senha} edge (pessoa);

% ---------------------
%        PERFIL
% ---------------------

  \node[attribute] (codigo-perfil) [below left = 5mm and 0 of perfil] {Código perfil} edge (perfil);
  \node[attribute] (tipo-perfil) [below right = 5mm and 0 of perfil] {Tipo perfil} edge (perfil);
  \node[attribute] (id-perfil) [below = 1cm of perfil] {\underline{Id}} edge (perfil);

% ---------------------
%       SERVICO
% ---------------------

  \node[attribute] (codigo-servico) [above = 1cm of servico] {Código serviço} edge (servico);
  \node[attribute] (descricao) [above right = 0 and 1cm of servico] {Descrição} edge (servico);
  \node[attribute] (servico-tipo) [below right = 0 and 1cm of servico] {Tipo serviço} edge (servico);
  \node[attribute] (id-servico) [below = 5mm of servico] {\underline{Id}} edge (servico);
 
% ---------------------
%       DOCENTE
% ---------------------

  \node[attribute] (docente-especialidade) [above right = 0 and 1cm of docente]{Especialidade} edge (docente);
  \node[attribute] (docente-funcao-tecnica) [above right = 1cm and 0 of docente] {Função técnica} edge (docente);
  \node[attribute] (id-docente) [above left = 5mm and 0 of docente] {\underline{Id}} edge (docente);
  
% ---------------------
%     FUNCIONARIO
% ---------------------

  \node[attribute] (funcionario-especialidade) [above left = 0 and 1cm of funcionario] {Especialidade} edge (funcionario);
  \node[attribute] (funcionario-funcao-tecnica) [below left = 0 and 1cm of funcionario] {Função técnica} edge (funcionario);
  \node[attribute] (funcionario-id) [below = 5mm of funcionario] {\underline{Id}} edge (funcionario);
  
% ---------------------
%        ALUNO
% ---------------------

  \node[attribute] (curso) [below left = 8mm and 0 of aluno] {Curso} edge (aluno);
  \node[attribute] (nota-ingresso) [below left = 0 and 1cm of aluno] {Nota ingresso} edge (aluno);
  \node[attribute] (id-aluno) [above right = 5mm of aluno] {\underline{Id}} edge (aluno);
  
% ---------------------
%      DISCIPLINA 
% ---------------------

  \node[attribute] (nome-disciplina) [above right = 4mm and 6mm of disciplina] {Nome} edge (disciplina);
  \node[attribute] (ementa) [right = 1cm of disciplina] {Ementa} edge (disciplina);
  \node[attribute] (codigo-disciplina) [below right = 1cm of disciplina] {Código disciplina} edge (disciplina);
  \node[attribute] (id-disciplina) [above left = 3mm and 7mm of disciplina] {\underline{Id}} edge (disciplina);
  \node[attribute] (data-criacao) [above = 8mm of disciplina] {Data criação} edge (disciplina);

% ---------------------
%      OFERECIMENTO 
% ---------------------

  \node[attribute] (nota-obtida) [below left = 1cm of rel-oferecimento] {Nota obtida} edge (rel-oferecimento);
  \node[attribute] (data-inicio) [below = 1.5cm of rel-oferecimento] {Data início} edge (rel-oferecimento);
  \node[attribute] (data-fim) [below right = 1cm of rel-oferecimento] {Data fim} edge (rel-oferecimento);
  
\end{tikzpicture}
    


\end{document}
